\documentclass{report}

\input{~/school/textemplates/notestemplate/afkortingen/preamble.tex}
\input{~/school/textemplates/notestemplate/afkortingen/macros.tex}
\input{~/school/textemplates/notestemplate/afkortingen/letterfonts.tex}

\title{\Huge{Extra fourier, laplace en z transforms}\\}
\author{\huge{Tibo Fordeyn}}
\date{}

\begin{document}

\maketitle


\newpage% or \cleardoublepage
% \pdfbookmark[<level>]{<title>}{<dest>}
\pdfbookmark[section]{\contentsname}{toc}
\tableofcontents
\pagebreak

\chapter{Fourier transforms}
fourier transforms zijn onderdeel van de laplace transforms.
Fourier transforms zoeken sinusoidale functies, laplace zoekt sinusoidale en exponenti"ele.
\[
\mathscr{F} \big[f(t)\big](\omega) &= \int_{-\infty}^{+ \infty} f(t)e^{-i \omega t} dt  
.\] 
\ex{}{
\[
y(t)&= \begin{cases}
	0, t<0 \\ 
	1, t>0 \\ 
\end{cases} 
.\] 	
\[
f(t)&= y(a-|t| ) 
.\] 
a is een constante functie en t is dus een absolutewaardefunctie. We zien dat deze functie overal nul is behalve wanneer:
\[
a-|t| >0 \iff |t| < a
.\] 
We noemen dit een blokgolf functie.
}
Aangezien de functie overal nul is behalve van min a naar a, integreren we niet over gans $\mathbb{R}$
\[
\mathscr{F} \big[f(t)\big](\omega) &= \int_{-a}^{a} 1\cdot e^{-i \omega t} dt  
.\] 
In het interval waar we wel over integreren is f 1, dus we kunnen gewoon 1 schrijven als f.
\[
\iff \mathscr{F} \big[f(t)\big](\omega) &= \left[ \frac{e^{-i \omega t} }{-i \omega } \right]  ^{a} _{-a} 
.\] 
\[
\iff \frac{e^{-i \omega a} }{-i \omega } - \frac{e^{i \omega a} }{-i \omega }
.\] 
\[
\iff \frac{1}{i \omega } \left[ e^{i \omega a} - e^{-i \omega a}  \right] 
.\] 
\[
\implies 2a \frac{\sin{(a\omega )}}{a\omega }
.\] 

\ex{}{
	Beschouw:
	\[
	f(t)&= e^{-|t| }  
	.\] 
	We willen de fourier transform van f.
}

\[
\mathscr{F} \big[e^{-|t| } \big](\omega) &= \int_{-\infty}^{+ \infty} e^{-|t| } \cdot e^{- i \omega t} dt   
.\] 
Een absolute waarde splits je altijd op.
\[
\mathscr{F} \big[f(t)\big](\omega) &= \int_{-\infty}^{0} e^{t} \cdot e^{-i \omega t} dt  + \int_{0}^{+ \infty} e^{-t} \cdot e^{-i \omega t} dt 
.\] 
Limiet nemen voor de oneigenlijke integraal.
\[
\iff \lim_{p \to \infty} \int_{-p}^{0} e^{t(1-i \omega )} dt + \int_{0}^{p} e^{-t(1+i\omega )} dt  
.\] 
\[
	\iff \lim_{p \to \infty}  \left[ \frac{e^{t(1-i \omega )} }{1-i \omega } \right] ^{0} _{- p} + \left[ \frac{e^{-t(1+i \omega )} }{-(1+i \omega )} \right] ^{p} _{0} 
.\] 
\[
\implies \frac{1}{1-i \omega } - \frac{1}{-(1+i\omega  )}
.\] 
\[
\implies \frac{2}{1-i\omega }
.\] 
\hline
\section{fourier voor afgeleiden van functies}%
\label{ssub:fourier voor afgeleiden van functies}

\\ 
\\ 
Je wilt waarschijnlijk eens de voorwaarden voor een functie om fourier transformeerbaar te zijn bekijken.
\section{AFGELEIDE FORMULE}%
\label{ssub:AFGELEIDE FORMULE}

\begin{enumerate}
	\item f continu over gans $\mathbb{R}$
	\item f' Stuksgewijs continu over elke periode
	\item f fourier transformeerbaar
	\item $\lim_{t \to \infty}f(t) &= 0 $
\end{enumerate}
dan geldt:
\[
\mathscr{F} \big[f'(t)\big](\omega) &= i \omega \hat{f}(\omega ) 
.\] 
\dfn{ Algemeen }{
	\textbf{Als}
	\begin{enumerate}
		\item $f^{(k-1)}$ continu is op gans $\mathbb{R}$ 
		\item $f^{(k)}$ stuksgewijs continu op elk interval
		\item $f^{(k-1)}$ moet fourier transformeerbaar 
		\item $\lim_{t \to \infty} f^{(k-1)} (t)&= 0 $
	\end{enumerate}
	\textbf{Dan}
	\[
	\mathscr{F} \big[f^{(k)} (t)\big](\omega) &= (i \omega )^{k}\cdot  \mathscr{F} \big[f(t)\big](\omega)  
	.\] 
}
\section{Verder toepassingen oplossen}%
\label{ssub:Verder toepassingen oplossen}
\ex{}{
	Beschouw:
	\[
	q(t)&= \begin{cases}
		e^{t}, t<0 \\ 
		-e^{-t}, t>0 \\  
	\end{cases} 
	.\] 
}
\[
\mathscr{F} \big[q(t)\big](\omega) &= \int_{-\infty}^{+ \infty} q(t)\cdot e^{-i \omega t} dt  
.\] 
\clm{Andere functie}{}{
	De functie $e^{-|t| }  $ hebben we alreeds fourier getransfomeerd.
	\[
	\mathscr{F} \big[e^{-|t| } \big](\omega) &= \frac{2}{1-i \omega }
	.\] 
	Stel $f(t)&= e^{-|t| }  $
	 \[
	 q(t)&= f'(t) 
	 .\] 
	 en
	 \[
	 \mathscr{F} \big[f^{(k)}(t)\big](\omega) &= (i\omega )^{k} \cdot \int_{-\infty}^{+ \infty} f(t)e^{-i \omega t} dt  
	 .\] 
}
\[
\iff \mathscr{F} \big[q(t)\big](\omega) &=  (i \omega )\cdot \int_{-\infty}^{+ \infty} f(t)e^{-i \omega t} dt  
.\] 
\[
\implies \mathscr{F} \big[q(t)\big](\omega) &= (i \omega) \cdot \frac{2}{1-i\omega } 
.\] 
\[
\implies Q(\omega )&= \frac{2 i \omega }{1-i\omega } 
.\] 
\ex{}{
Beschouw:
\[
y''+y&= 0 
.\] 
}
\[
\mathscr{F} \big[y"\big](\omega) + \mathscr{F} \big[y\big](\omega) &= \mathscr{F} \big[0\big](\omega)  
.\] 
\[
	(i\omega )^2\mathscr{F} \big[y\big](\omega) + \mathscr{F} \big[y\big](\omega) &= 0 
.\] 
\[
\mathscr{F} \big[y\big](\omega) &= 0 
.\] 
\[
y&= 0 
.\] 
we vinden geen sinus en cosinus als oplossingen, want sinus en cosinus zijn niet foutier transformeerbaar. (de voorwaarde $\lim_{x \to \infty} f(x)&= 0 $ is niet voldaan.)








\end{document}
Footer

