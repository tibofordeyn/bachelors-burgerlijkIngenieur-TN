\documentclass{report}

\input{~/school/textemplates/notestemplate/afkortingen/preamble.tex}
\input{~/school/textemplates/notestemplate/afkortingen/macros.tex}
\input{~/school/textemplates/notestemplate/afkortingen/letterfonts.tex}

\title{\Huge{Scheikunde}\\Eerste Semester}
\author{\huge{Fordeyn Tibo}}
\date{}

\begin{document}

\maketitle


\newpage% or \cleardoublepage
% \pdfbookmark[<level>]{<title>}{<dest>}
\pdfbookmark[section]{\contentsname}{toc}
\tableofcontents
\pagebreak

\chapter{Fundamentele kennis, perfecct vanbuiten kennen voor examen}




\chapter{Herhaling middelbaar}
Ik ga hier zo snel mogelijk over.
\section{Materie, fundamenele begrippen en definities}

\begin{itemize}
	\item Etensieve eigenschappen: Massa en volume. Hangen af van hoeveelheid materie
	\item Intensieve eigenschappen: dichtheid. Onafhankelijk van de hoeveelheid materie 
\end{itemize}
We onderscheiden
\begin{itemize}
	\item Fysische eigenschappen: Eigenschappen die een stof op zichzelf vertoont zonder op welke manier dan ook te reageren met een andere stof. 
		\begin{itemize}
			\item kleur
			\item smeltpunt
			\item kookpunt
			\item elektrische geleidbaarheid
			\item dichtheid
		\end{itemize}
		Een voorbeeld van een fysische veradering is het smelten van ijs.
	\item chemische eigenschappen hebben te maken met hoe de stof wijzigt in reactie met andere stoffen.
		\begin{itemize}
			\item Ontvlambaarheid of corrosiviteit
			\item reactiviteit met water
		\end{itemize}
		Een verandering in de chemische eigenschappen van een stof is een chemische reactie en die verandert ook de fysische eigenschappen.
\end{itemize}

Wanneer de samenstellende deeltjes (ionen, atomen of moleculen) gelijk gerangschikt zijn in de ruimte spreekt men van \textbf{kristallijne vaste stoffen} dit zijn dus stoffen met een kristalstructuur.

\\ Wanneer zo'n schikking op macroscopische schaal plaatsvindt praten we over één-kristallen of monokristallijn.
Een groot aantal monokristallijnen zijn \textbf{kristallieten}. Zorn materialen worden polykristallijn genoemd, de meeste metalen zijn polykristallijn.

stoffen zonder kristallijne structuur noemt men amorf.
\\ De \textbf{viscositeit} van vloeistof is de maat voor de weerstand van een vloeistof tegen stroming.

\\ \textbf{Fluidum} heeft onder vaste temperatuur en druk een welbepaalde massa en volume, maar geen vaste vorm.

verschil tussen koken en verdampen:
\begin{itemize}
	\item Koken gebeurt bij een bepaalde temperatuur (kookpunt), en \textbf{bij de hele vloeistof}.
	\item Verdampen gebeurt bij elke temperatuur, maar \textbf{enkel aan het oppervlak}.
\end{itemize}


\subsubsection{Fasetoestanden}%
\label{ssub:Fasetoestanden}

\begin{tabular}{ |p{3cm}||p{4cm}|p{3cm}|p{3cm}|  }
\hline
\multicolumn{2}{|c|}{Tabel 1: Fasetoestanden, 1 bar, T = 25°C} \\
\hline
fasetoestanden& Kookpunten\\
\hline
Gas   & $T_{kook} < 25°C$   \\
Vloeibaar& $T_{smelt} < 25 °C < T_{kook} $    \\
Vast & $T_{smelt} > 25 °C $\\
\hline
\end{tabular}

\section{Het periodiek systeem van de elementen}

\begin{figure}[htpb]
	\centering
	\includegraphics[width=0.8\textwidth]{figures/pseGroepen.jpeg}
	\caption{figures/pseGroepen.jpeg}
	\label{fig:-figures-pseGroepen-jpeg}
\end{figure}
 Dus je hebt de A en de B groepen, en de periodes. De A groep omvatten de hoofgroepen of representatieve elementen. De B groep omvat de transitiemetalen, waarbij de lanthemiden en actimiden de binnenste transitiemetalen worden genoemd.


 Je weet ook nog uit het eerste middelbaar ofzo dat enkel de edelgassen als ongebonden atomen voorkomen.

 \subsubsection{Belangrijke groepen in het pse}%
 \label{ssub:Belangrijke groepen in het pse}
 Zorg dat je hier goede anki kaartjes van maakt, wan ik ken dit echt nog niet vanbuiten.

 \begin{enumerate}
 	\item De \textbf{alkalimetalen}, \textbf{GROEP 1A}
bestaat uit
\begin{itemize}
	\item Li
	\item Na
	\item K
	\item Rb (rubidium)
	\item Cs (cesium)
	\item Fr (franscium)
\end{itemize}
Dit zijn allemaal \textbf{glimmende, zachte metalen}, en met uitzondering van Fr dat bij een druk van 1 bar een smeltpunt heeft van 21°C, zijn alkalimetalen vaste stoffen.
Ze reageren vaak snel en heftig met water tot vorming van sterk basische (alkalische) producten.
waterstof (H) wordt in deze groep ondergebracht ondanks de vele verschilen met andree elementen en de rerden darvoor wordt besproke in H4.
\item \textbf{Aardkalimetalen}, \textbf{GROEP 2A} 
	bestaat uit
	\begin{itemize}
		\item elementen die je niet vanbuiten moet kennen ofzo, ga stoppen met opschrijven want staat gewoon op pse lees gwn af
	\end{itemize}
	Dit zijn glanzende zilverkleurige metalen. 
ze zijn minder reactief en harder dan de elementen in 1A. Ze hebben een hoger smeltpunt en het zijn allemaal vaste stoffen.


\item de \textbf{halogenen}, \textbf{GROEP 7A}
	wegens hoge reactiviteit komen deze voor in discrete datomaire molecule $(F_{2} )$ of in combinatie met andere elementen. Dit is de enige groep die de elementen in drie fasetoestanden bevat bij 1 bar druk en 25°C. Fluor en Chloor zijn gassen, broom is een vloeistof en jodium is een vaste stof.
\item de edelgassen 8A
	reukloos kleurloos weinig reactief.
	Merk op dat ze voorkomen als di-atomaire gassen bij 25°C en atmosfeerdruk.
 \end{enumerate}

 \cor{Semi-metalen}{
	 De zeven uit 9 elementen grenzend aan de niet metalen worden semi-metalen geoemd. Lv en Tv niet megerekend. 
 }

 
 Diamant en grafiet zijn \textbf{allotropen} van koolstof. Allotropen van een molecule zijn verschillende vormen van die moleculen.
 
\section{De atomaire visie op materie}






\section{De hedendaagse atoomtheorie}

\section{Chemische verbindingen en chemische formules}

\subsection{ionaire bindingen}
Kunnen niet worden voorgested door de empirische-formule-eenheid(EFE), want doordat het aantal protonen en elektronen niet gelijk zijn voor elk element binnen een molecuul, zijn er eigenlijk geen volledige atomen aanwezig. De sto"echiometrische coefficienten zijn $\notin \mathbb{Z}$ 
\subsection{covalente bindingen}
Men onderscheidt twee soorten van covalente bindingen; moleculen en covalente netweren.
De \textbf{kationen} zijn de positief geladen deeltjes en de \textbf{anionen} de negatief geladenen.

\begin{itemize}
	\item De samenstellende stoffen kunnen mono-atomair zijn zoals $Na^{+} $
	\item poly-atomair zoals $\left( NH_{4}  \right) ^{+} $
\end{itemize}
Ladingen van ionen kan voorgesteld worden als een optelling van eenheidsladingen.

\[
\text{elektroneutraliteitsprincipe } \iff \sum_{i}^{ } q_{i} N_{i,J} &= 0 
.\] 

\subsubsection{Moleculen}%
\label{ssub:Metalen}
Het aantal moleculen aanwezig in zon binding issteeds een groot geheel getal. Dit in tegenstelling tot de ionaire bindingen waarbij de sto"echiometrische co"effici"enten niet geheel zijn.

Ze komen voor in alle agregatietoestanden.
Men kan moleculen verder opdelen:
\begin{itemize}
	\item homonucleaire diatomaire moleculen ($A_{z} $)
	\item heteromoleculaire diatomaire moleculen ($A B $)
	\item binaire moleculen ($A_{n} B_{m} $)
	\item poly-atomaire moleculen
\end{itemize}


	\begin{figure}[htpb]
		\centering
		\includegraphics[width=0.8\textwidth]{figures/tabelCovalentIonair.jpeg}
	\end{figure}

	relatieve atoommassa:
	\[
	RAM_{J} &= \sum_{X}^{ } N_{X,J} RAM_{X}  
	.\] 

	bijvoorbeeld de relatieve atoommassa van gluccose 
	\[
	R M_{C_{6} H_{12} O_{6}  } &= N_{C_{6} H_{12} O_{6}    } RAM_{C} + N_{C_{6}       H_{12} O_{6}    } RAM_{H} + N_{C_{6}       H_{12} O_{6}    } RAM_{O}  
	.\] 
	\[
	\implies 180.1572 amu
	.\] 

	Getal van avogadro:
	\[
	6.022137 \cdot 10^{23} 
	.\] 
\subsubsection{Experimenteel bepalen van massas}%
\label{ssub:Experimenteel bepalen van massas}
Een techniek regelmatig gebruikt om de EFE van een verbinding te bepalen is door het de laten reageren met dioxide, eigenlijk dus het laten verbranden van de verbinding.


\chapter{Mengsels en oplossingen}
kwalitatieve samenstelling van een mengsel vermeldt het element val elk van de zuivere stoffen aanwezig in het mengsel.
De kwantitatieve samenstelling van een mengsel vermeldt hiernaast ook de hoeveelheid van elk van de zuivere stoffen aanwezig is.

\section{Samenstellingen en mengsels}
Die gegeven hoeveelheid bij kwantitative samenstellingen wordt weergegeven adhv massafracties of soms ook adhv massapercentages.

\[
\gamma_{m,J} &= \frac{m_{J} }{m_{tot} } &= \frac{m_{J} }{\sum_{J}^{ } m_{J} }  &= \frac{n_{J} MM_{j} }{\sum_{J}^{ } n_{j} MM_{j} } 
.\] 
Is de massafractie van verbinding J in een mengsel.

Van een mengsel kan \textbf{geen }molaire massa gedefinieerd worden, aangezien een mengsel geen constante samenstelling heeft.


\subsubsection{massa mengesel}%
\label{ssub:massa mengesel}
De totale massa van een mengsel kan geschreven worden als de som van elke component:
\[
m_{tot} &= \sum_{i}^{ } m_{i,tot}  \equiv \sum_{i}^{ } \gamma_{m,i,tot} m_{tot}  
.\] 
\cor{$\gamma$}{
Dit is een decimaal getal tussen nul en één voor de didelijkheid.	
}
\section{Gasmengsels}
Gasdruk ontstaat als gevolg van botsingen van de gasdeeltjes met wanden van het vat.
De si eeinheden voor druk: Pascal(PA), bar, atmosfeer(atm) of mmHg. Zeer lage drukken kunnen soms worden uitgedrukt in torr.
\dfn{ De wet van Boyle }{
	Voor een gegeven massa van een gas dat opgesloten is in een vat met variabel volume V, een vermindering van het volume proportioneel gepaard gaat met stijging van de druk P, indien de termperatuur constant blijft.
	\[
	P_{1} V_{1} &= P_{2} V_{2}  
	.\] 
}
Gay Lussac en Charles voerden experimenten uit om het verband tussen gasvolumes en termperatuur te bepalen.
\dfn{ Gay Lussac en Charles }{
	Voor een gegeven massa van een gas dat opgesloten zit in een vat met variabel volume, toename van het volume gepaard gaat met een proportionele toename van de termperatuur als druk constant gehouden wordt.
	\[
	\frac{V_{1} }{T_{1}  }&= \frac{V_{2} }{T_{2} }
	.\] 
}
Beide wetten combineren geeft:
\[
\alpha T &= PV \text{ of } \alpha &= \frac{P_{1} V_{1} }{T_{1}   } &= \frac{P_{2} V_{2} }{T_{2} }   
.\] 
ideale gaswet:
\[
PV &= nRT 
.\] 
Met R de gasconstante.
\[
R &= \frac{PV}{nT} 
.\] 
Een ideaal gas voldoet hieraan onder alle condities.

Molair volume:
\[
V_{m} &= \frac{nRT}{P} 
.\] 
De normaalomstandigheden bij $0°C$ en 1 bar worden als referentiecondities gebruikt voro eht rapporteren.

\[
MM &= \frac{dRT}{P} 
.\] 
waar $d&= \frac{m}{V} $

De totaaldruk van een mengsel:
\[
P_{tot} &= \sum_{i}^{ } \frac{n_{i} RT}{V} 
.\] 
en dus de partieeldruk
\[
P_{i} &= \frac{n_{i} RT}{V} 
.\] 
\[
P_{i} &= \gamma_{n,i} P_{tot}  
.\] 
\section{toestandswijzigingen}
\[
R &= \frac{P_{1} V_{1} }{nT_{1} } &= \frac{P_{2} V_{2} }{nT_{2} } 
.\] 
met temperatuur in kelvin, druk in atm, volume in liter. 

Merk op dat:
\[
V_{2} &= V_{1} \frac{P_{1} T_{2} }{P_{2} T_{1} } 
.\] 







\section{NPGE 1}










\chapter{Chemische reacties}
\[
\text{atoombalans } \iff \sum_{i}^{react} N_{i,R }  &= \sum_{i}^{prod}  N_{i,P}  
.\] 
\ex{}{
	Beschouw:
	\[
	H_{2} +O_{2} \to H_{2} O
	.\] 
	Merk op dat er geen atoombalans is.
}
Iets als
\[
H_{2} +O_{2} \to H_{2} O + O
.\] 
schrijven zou fout zijn, wan zuurstof komt niet apart voor.

De juiste manier om dit op te lossen;
\[
2H_{2} + O_{2} \to  2 H_{2} O
.\] 
De getallen voor de stof worden \textbf{stochiometrische coefficientenici"enten } genoemd. De kleinst mogelijke gehele getallen. Daar is één uitzondering op en die komt straks.

\ex{}{
	Beschouw de ontbindingsreactie van nitroglycerine:
	\[
	\alpha C_{3} H_{5} N_{3} O_{9} \to xN_{2} +yCO_{2} +zO_{2} + wH_{2} O
	.\] 
}
Er is een manier om dit eenvoudig op te lossen op een manier die altijd werkt.

zet om naar formele algebrarische vergelijking. 
Je krijgt een stelsel voor elk elemnt end an zal de opossing vanzel duidelijk worden.
\cor{gehele getallen}{
Omdat het gehele getallen moeten zijn kun je eigenlik gewoon $\alpha&= 1   $   	stellen, kijken wat je krijgt en alles vermenivuldigen met een geheel getal.
} 

\[
\text{ladingbalans } \sum_{i}^{react} q_{i,R} &= \sum_{i}^{prod } q_{i,P}  
.\] 
\section{Wat betekent een chemische reactievergelijking}

\section{Soorten reacties}

\subsection{Neerslagreacties}
\subsection{Redoxreacties}
Om een redoxreactie in evenwicht te brengen splits je op in twee \textbf{halfreacties}.
\ex{}{
	Bij oplossen van vast arseensulfide in een waterige oplossing van kaliumchloraat wordt arseenzuur ($H_{2} AsO_{4} $), kaliumsulfaat en kaliumchloride gevormd.
	De bedoeling is dat de redoxreactie in evenwicht gebracht wordt.
}
In de eerste zin krijg je info over de reactanten.
\[
\text{Reactanten } \iff \begin{cases}
	As_{2} S_{3} _{(s)} \\ 
KClO_{3} _{(aq)} 
\end{cases}
.\] 
\[
\text{Producten } \iff \begin{cases}
	H_{3} AsO_{4} \\ 
	KSO_{4} \\ 
	KCl \\ 
\end{cases} 
.\] 
\textbf{Wat weten we over de reactanten?}
\\ Merk op dat vast arseensulfide een covalente binding is, dus daar komt geen ion aan te pas.
\\ Kaliumchloraat is een ionaire verbinding, dat wil zeggen dat we gehydrateerde $K^{+} $ ionen en $ClO_{3} ^{-} $ ionen krijgen.
\textbf{Wat weten we over de producten?}

\\ Arseenzuur is een zwak zuur dus in water komt dit voornamelijk voor in de vorm van moleculen.
\\ Kaliumsulfaat is een ionaire binding, je krijgt gehydrateerde $K^{+} $ en $ClO_{3} ^{-}  $ ionen.
kaliumchloride is zeker een ionaire binding dus je krijgt weer die gehydrateerde ionen.



\\ Deze twee moeten we in halfreacties opschrijven zegmar.
\begin{itemize}
	\item je bekijkt waar de delen van de reactie naartoe gaan.
		\[
		ClO_{3} ^{-} \to Cl^{-} 
		.\] 
		\[
		As_{2} S_{3} \to H_{3} AsO_{4} + SO_{4} ^{2-} 
		.\] 
	\item Nu moeten we ze in evenwicht brengen, breng eerst alle aomen behalve \textbf{zuurstof en waterstof} in evenwicht
		\[
		\begin{cases}
			ClO_{3} ^{-} \to Cl^{-} \\ 
			As_{2} S_{3} \to 2 H_{3} AsO_{4} + 3SO_{4} ^{2-} 
		\end{cases}
		.\] 
	\item Nu in evenwicht brenen voor zuurstof door toevoegen van atermoleculen langs de kant war zuursstofmoleculen te kort zijn.
		\[
		\begin{cases}
		ClO_{3} ^{-} \to Cl^{-} +3H_{2} 0 \\ 
		As_{2} S_{3} +8H_{2} 0 \to 2H_{3} AsO_{4} + 3SO_{4} ^{2-} 
		\end{cases}
		.\] 
	\item Ten slotte brengen we in evenwicht voor watersotf
		\[
		\begin{cases}
		ClO_{3} ^{-} + 6H \to Cl^{-} + 3H_{2} O \\ 
		As_{2} S_{3} +12H_{2} O \to 2H_{3} AsO_{4} +3SO_{4} ^{2-} +18H
		\end{cases}
		.\] 
	\item en dan ten slotte niet vergeten om elektronen weg te nemen of erbij te tellen, ik heb er nu geen zin in, maar niet vergeten.
	\item het aantal elektronen bij de reactanten moet gelijk zijn aan het aantal bij de producten, anders is de ractie ook niet in evenwicht.
		
\end{itemize}

\cor{Hoe zorg je dat die aantallen elektronen gelijk zijn aan elkaar?}{
	Je vermenigvuldigd de erste halreactie met het aantal elektronen van de tweede halfreactie en de tweede met die van de eerste.
}
\begin{itemize}
	\item dan moet je ze bij elkaar brengen, je schrapt de dingen di gelijk zijn en zorgt dat er niet gedeeld kan worde door een geheel getal ove de hele vergelijking. 
	\item dan controleren en je bent klaar.
\end{itemize}

\cor{zwakke en sterke zuren}{
	Een zwak zuur komt in waterige oplossing voornamelijk voor as molecule, een sterk zuur komt eerder voor als $H_{3} O^{+}  $ en een geconjugeerde base.
}











\section{titratie}
een techniek om de concentratie van een oplossing te bepalen door het in reactie te brengen met iets anders.

We beginnen met
\[
aA+bB \to cC+dD
.\] 
We zorgen dat er exact evenveel A wegreageert als dat er origineel inzat, dus de hoeveelheid B die is toegevoegd moet gelijk zijn aan die van A om te zorgen dat het kan wegreageren.
\[
\frac{n_{A,o} }{a}&= \frac{n_{B,toegevoegd} }{b} 
.\] 
dat betekent dat we het equivalentiepunt bereikken.
Het equivalentiepunt kan duidelijk gemaakt worden door ee indicator. Afhankelijk van het soort tiratie kies je een andere indicator.
\section{Zuur-base titratie}
\[
zuur+base \to zout
.\] 
\ex{}{
	Beschouw:
	\[
	2HNO_{3} _{(aq)} + Ba(OH)_{2(aq)} \to Ba(NO_{3} )_{2(aq)} + 2H_{2} O_{(l)} 
	.\] 
}



\section{NPGE2}



\chapter{atoomstructuur}


\section{atoomstraal}
\begin{enumerate}
	\item covalente straal: helft van afstand tussen kernen van twee covalent gebonden atomen.
	\item van der Waalsstraal: helft van afstand van twee identieke niet gebonden atomen.
\end{enumerate}

\section{ionisatie energie}
energie nodig om van het atoom de eerste kation te maken, noemt mn de eerste ionisatie energie
tweede ion toevoegen is tweede ionisatie energie, maar natuurlijk bij de tweede ionisatie energie verwijder je een elektron van hetkation, en vorm je een tweewaardige.


als het anion van a stabieler is dan a, dan zeggen we dat a affiniteit om het op te nemen.
de energie is lager dan nul, als de energie groter is dan nul dan heeft heeft a geen affiniteit om een elektron op te nemen.














\end{document}
Footer

