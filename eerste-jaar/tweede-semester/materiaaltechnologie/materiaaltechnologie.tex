\documentclass{report}

\input{~/school/textemplates/notestemplate/afkortingen/preamble.tex}
\input{~/school/textemplates/notestemplate/afkortingen/macros.tex}
\input{~/school/textemplates/notestemplate/afkortingen/letterfonts.tex}

\title{\Huge{Materiaaltechnologie}\\Tweede semester}
\author{\huge{Fordeyn Tibo}}
\date{}

\begin{document}

\maketitle


\newpage% or \cleardoublepage
% \pdfbookmark[<level>]{<title>}{<dest>}
\pdfbookmark[section]{\contentsname}{toc}
\tableofcontents
\pagebreak

\chapter{Atomic structure and interatomic bonding}
\section{Atomic structure}
Energietoestand uitgedrukt in elektronvolt, geeft aan hoeveel energie er nodig is om een elektron weg te trekken uit een schil.
\[
1eV&= 1,602 \cdot 10^{19} J
.\] 
Elektron volt zegt eigenlijk hoeveel energieverandering er is van een $e^{-}$ als het een weg aflegt van $1V$

Naarmate je een hoger subniveau bekijkt, zal de energietoestand stijgen. De energietoestand van een p subniveau is groter.

\end{document}
Footer

