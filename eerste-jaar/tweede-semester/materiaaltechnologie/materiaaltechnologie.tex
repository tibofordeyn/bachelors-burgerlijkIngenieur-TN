\documentclass{report}

\input{~/school/textemplates/notestemplate/afkortingen/preamble.tex}
\input{~/school/textemplates/notestemplate/afkortingen/macros.tex}
\input{~/school/textemplates/notestemplate/afkortingen/letterfonts.tex}

\title{\Huge{Materiaaltechnologie}\\Tweede semester}
\author{\huge{Fordeyn Tibo}}
\date{}

\begin{document}

\maketitle


\newpage% or \cleardoublepage
% \pdfbookmark[<level>]{<title>}{<dest>}
\pdfbookmark[section]{\contentsname}{toc}
\tableofcontents
\pagebreak

\chapter{Atomic structure and interatomic bonding}
\section{Atomic structure}
Energietoestand uitgedrukt in elektronvolt, geeft aan hoeveel energie er nodig is om een elektron weg te trekken uit een schil.
\[
1eV&= 1,602 \cdot 10^{19} J
.\] 
Elektron volt zegt eigenlijk hoeveel energieverandering er is van een $e^{-}$ als het een weg aflegt van $1V$

Naarmate je een hoger subniveau bekijkt, zal de energietoestand stijgen. De energietoestand van een p subniveau is groter.


\section{Bindingsenergie en}
Net energie $F_{N}  $ tussen twee atomen is de som tussen de aantrekkende en repulsieve kracht.
\[
F_{N} &= F_{A} + F_{R} 
.\] 
\clm{Potentiële energie}{}{
	\[
	E_{p} &= \int_{ }^{ } F dr 
	.\] 
}

\[
E_{N}  &= \int_{r}^{+ \infty} F_{N} dr &= \int_{r}^{+ \infty} F_{A} + F_{R} dr  
.\] 
\[
\implies E_{A} + E_{R} &= E_{N} 
.\] 

dus ook
\[
F_{N} &= \frac{dE_{A} }{dr} + \frac{dE_{R} }{dr}
.\] 

\chapter{The structure of crystalline Solids}
\section{fundamentele concepten}
Mogelijke schikkingen van atomen in een vaste stof:
\begin{itemize}
	\item kristallijn of niet-kristallijn
	\item kristalsystemen en structuern (metalen)
	\item polymorfisme en allotrope transformaties
	\item kristallografische richtingen, vlakken
	\item aotmaire schikkingen in 1D,2D,3D pakkingen
	\item éénkristal, polykristallijn, amorfe materialen

\end{itemize}

Een kristallijn is een zichzelf herhalende of periodieke herhaling van atomen over 'lange' afstanden
\begin{itemize}
	\item dt zijn alle metalen meestal keramieke, soms polymeren, onder normale stollinsomstandighden
	\item materiaal vertoont orde op lange afstand na stollen
	\item repititieve 3D structuur
\end{itemize}
worden vaak gevisualiseerd via harde sferen model.
De kristalstructuur is de wjze waarop atomen geordend zijn.



\end{document}
Footer

