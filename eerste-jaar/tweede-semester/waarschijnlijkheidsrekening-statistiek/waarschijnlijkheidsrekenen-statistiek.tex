\documentclass{report}

\input{~/school/textemplates/notestemplate/afkortingen/preamble.tex}
\input{~/school/textemplates/notestemplate/afkortingen/macros.tex}
\input{~/school/textemplates/notestemplate/afkortingen/letterfonts.tex}

\title{\Huge{Waarschijnleikheid en statistiek}\\Tweede semester}
\author{\huge{Fordeyn Tibo}}
\date{}

\begin{document}

\maketitle


\newpage% or \cleardoublepage
% \pdfbookmark[<level>]{<title>}{<dest>}
\pdfbookmark[section]{\contentsname}{toc}
\tableofcontents
\pagebreak

\chapter{De taal}
Uitspraken die zich vertalen naar wiskunde.

Alles dat niet in A zit.
\[
\text{co}(A) &= \left\{ \forall x \notin A \right\}  
.\] 


Als A gebeurt dan impliceert dit B. Een implicatie.
\[
A \subset B
.\] 

\section{waarscheinlijkheidsmaat}
Een functie P die een gebeurtenis als input neemt en een getal als output geeft.
\[
P: \left[ P(B)\to R: A \to  P(A) \right] 
.\] 
Stel A is de zekere gebeurtenis:
\[
P(A)&= 1 
.\] 


Eigenschappen van belang:
\begin{itemize}
	\item De waarschij
	\item  $P(\text{co}(A))+P(A)&= 1 $
	\item 
\end{itemize}













\end{document}
Footer

